\documentclass[]{article}

%opening

\title{Introduci\'on al \'Algebra Lineal (Strang): Revisi\'on de Ideas Claves}

\author{Ciencias de la Computaci\'on V}

\usepackage[hidelinks]{hyperref}
\usepackage{cancel}
\usepackage{pst-node}
\usepackage{textcomp}
\usepackage{amssymb}
\usepackage{wasysym}
\usepackage{multirow} 
\usepackage[T1]{fontenc}

\usepackage[spanish]{babel}
\usepackage{graphicx}
\usepackage{epstopdf}
\usepackage{float}
\usepackage{mathrsfs}
\epstopdfsetup{outdir=./} 
\selectlanguage{spanish}
\usepackage[utf8]{inputenc}


\newtheorem{def1}{Definición}
\newtheorem{definicion}[def1]{Definición}

                          

\begin{document}
	
\maketitle

\section{Capitulo 1 Introducci\'on a vectores}

\subsection{Vectores y combinaciones lineales}

\begin{itemize}
	
	\item Un vector $ v $ en el espacio de dos dimensiones tiene dos componentes $ v_{1} $ y $ v_{2} $.  
	
	\item $ v+w=(v_{1}+w_{1},\ v_{2}+w_{2}) $ y $ cv=(cv_{1}, cv_{2}) $  se halla un componente por vez. 
	
	\item La combinación lineal de tres vectores $ u $, $ v $ y $ w $ es $ cu+cv+cw $.
	
	\item Se toma todas las combinaciones lineales de $ u $ o $ u $ y $ v $ o $ u,\ v,\ w $. En tres dimensiones estas combinaciones suelen ocupar una linea, luego un plano y todo el espacio $ \mathbb{R}^{3} $.  
	
\end{itemize}

\subsection{Longitud y producto punto}

\begin{itemize}
	
	\item El producto punto $ v\cdot w $ multiplica cada componente $ v_{i} $ por $ w_{i} $ y adiciona todo $ v_{i}w_{i} $. 
	
	\item La longitud $ \| v \| $ de un vector es la ra\'iz cuadrada de $ v\cdot v $. 
	
	\item $ u=v/ \| v \| $ es un vector unitario. Su longitud es 1. 
	
	\item El producto punto es $ v\cdot w = 0 $ cuando el vector $ v $ y $ w $ son perpendiculares. 
	
	\item El coseno de $ \Theta $ (el angulo entre cualquier $ v $ y $ w $ distinto de cero) nunca excede 1: 
	\[
	cos \Theta = \frac{v\cdot w}{\| v\| \|w \|} 
	\] 
	\textbf{Desigualdad de Schwarz}
	\[
	|v\cdot w|\leq \| v\| \|w \|
	\]
	El problema 21 produce la desigualdad del triangulo  $ \| v + w \| \leq \| v\| + \|w \| $
	
\end{itemize}

\subsection{Matrices}

\begin{itemize}
	
	\item \textbf{Matrix times vector}: $ Ax = $ combinaci\'on de las columnas de $ A $.  
	
	\item La soluci\'on a $ Ax=b $ es $ x=A^{-1}b $, cuando $ A $ es una matriz invertible. 
	
	\item La matriz de diferencia $ A $ se invierte por la matriz suma $ S0A^{-1} $. 
	
	\item Las matrices c\'iclicas son no invertibles. Sus tres columnas se encuentran en el mismo plano. Estas columnas dependientes se adhieren al vector cero. $ Cx=0 $ tiene muchas soluciones.  
	
\end{itemize}

\section{Capitulo 2 Solución de ecuaciones lineales}

\subsection{Vectores y ecuaciones lineales}

\begin{itemize}
	\item Las operaciones básicas sobre vectores son la multiplicación por escalar y la adición de vectores.
	\item La utilización de la multiplicación por escalar y la suma conforman una combinación lineal. Ej: cv+dw.
	\item La matriz de multiplicación Ax, puede ser computada utilizando productos puntos, una fila a la vez. Pero Ax debe ser entendido como una combinación de columnas de A.
	\item AX=b, ``b'' es una combinación lineal de las columnas de la matriz de A.
	\item Cada ecuación en Ax=b da una línea (n=2) o un plano (n=3) o un hiperplano cuando (n>3)
\end{itemize}

\subsection{La idea de la eliminaci\'on}

\begin{itemize}
	\item Un sistema Ax=b se transforma en un sistema triangular Ux=c, luego de la eliminación.
	\item Se sustrae $l_{ij}=\frac{entrada-eliminar-en-fila-i}{pivot-en-fila-j}$
	\item Un vector cero en una posición pivot puede ser reparado si hay un no cero abajo.
	\item Un sistema triangular superior es resuelto por sustitución hacia atrás.
	\item Cuando un corte es permanente, el sistema puede no tener solución o muchas soluciones.
\end{itemize}

\subsection{Eliminaci\'on usando matrices}

\begin{itemize}
	\item $Ax=x_{1}$ veces las columnas $1+\cdots+x_{n}$ veces la columna $n$. $\sum_{j=1}^n a_{ij}x_{j}$.
	\item Multiplicando $Ax=b$ por $E_{21}$ sustrae un múltiplo $l_{21}$ de la ecuación 1 de la ecuación 2. El número $-l_{21}$ es la entrada $(2,1)$ de la matriz de eliminación $E_{21}$.
	\item Para una matriz aumentada $[A b]$, el paso de eliminación da $[E_{21}A E_{21} b]$.
	\item Cuando se multiplica A por cualquier matriz B esta multiplica cada columna de B por separado.
\end{itemize}

\subsection{Reglas para operaciones de metrices}

\begin{itemize}
	\item Las entradas (i,j) de AB es (fila de i de A) . (columna j de B).
	\item Una matriz de m x n veces por una matriz n x p usa mnp multiplicaciones separadas.
	\item AB es también la suma de estas matrices: (columna de j de A) veces (fila de j de B).
	\item La multiplicación por bloque es permitida cuando las figuras coinciden correctamente.
	\item La eliminación por bloque produce el complemento $D-CA^{-1}B$.
\end{itemize}

\subsection{Matriz inversa}

\begin{itemize}
	
	\item La matriz inversa esta dada por $ AA^{-1} = I $ y   $ A^{-1}A = I $. 
	
	\item $ A $ es invertible si y solo si tiene $ n $ pivots (con intercambios de filas permitios). 
	
	\item Si $ Ax=0 $ para un vector $ x $ distinto de $ 0 $, entonces $ A $ no es invertible. 
	
	\item La inversa de $ AB $ es el producto de $ B^{-1}A^{-1} $. Y $ (ABC)^{-1}=C^{-1}B^{-1}A^{-1} $. 
	
	\item El m\'etodo Gauss-Jordan resuelve $ AA^{-1}=I $ para encontrar las $ n $ columnas de $ A^{-1} $. La matriz aumentada $ [AI] $ esta reducida a la fina $ [I A^{-1}] $.  
	 
	
\end{itemize}

\subsection{Factorizaci\'on A=LU}

\begin{itemize}
	\item La eliminación gaussiana(sin cambio de filas) factorea A en L veces U.
	\item La matriz triangular inferior L contiene los números $l_{ij}$ que multiplica las filas pivots, desde A produciendo U. El producto LU agrega las filas para recuperar A.
	\item En el lado derecho se resuelve $Lc=b$ (forward) y$Ux=c$ (backward).
	\item \textbf{Factor:} Existe un $\frac{1}{3}(n^3-n)$ multiplicaciones y substracciones del lado izquierdo.
	\item \textbf{Solución:}Existe un $n^2$ multiplicaciones y substracciones en el lado derecho.
	Para una matriz banda, cambiar $\frac{1}{3}n^3$ a $nw^{2}$ y cambia $n^2$ a $2wn$.
\end{itemize}

\subsection{Transpuesta}

La transpuesta coloca las filas de A en columnas de $A^{T}$. Entonces ($A^{T}_{ij}=A_{ji}$).
Las transpuesta de AB es $B^{T}A^{T}$. La transpuesta de $A^{-1}$ es la inversa de $A^{T}$.
El producto punto es $x \cdot y=x^{T}y$. Entonces $(Ax)^{T}y$ es igual al producto punto de $x^{T}(A^{T}y)$.
Cuando una matriz es simétrica $(A^{T}=A)$, su factorización $LDU$ es simétrica: $A=L D L^{T}$.
Una matriz de permutación P tiene uno en una fila y columna y $P^{T}=P^{-1}$.
Existen $n!$ matrices de permutaciones del tamaño de $n$. Mitad par y mitad impar.
Si A es invertible, entonces la permutación P reordenando sus filas es $PA=LU$.

\section{Capitulo 3 Vectores y Subespacios}

\textbf{Autor:} Luis Esteban Mart\'inez Lailla stban06@gmail.com, Facultad Polit\'ecnica - Universidad Nacional de Asunci\'on.

\subsection{3.1 - Espacios de Vectores}
\begin{definicion}
	El espacio $R^n$ consiste en todos los vectores columnas con $n$ componentes.
\end{definicion}

$M \Rightarrow$ El espacio vector de todas las matrices 2x2.
$F \Rightarrow$ El espacio vector de todas las funciones reales $(f(x))$.
$Z \Rightarrow$ El espacio vector que consiste en solo el vector cero.

\begin{definicion}
	Un subespacio de un espacio vector es un grupo de vectores (incluyendo 0) que satisfaga 2 requerimientos:
	\begin{enumerate}
		\item $V + W $ est\'a en el subespacio.
		\item $cV$ est\'a en el subespacio.
	\end{enumerate}
\end{definicion}
Un subespacio que contiene V y W, debe contener todas las combinaciones lineales $cV+dW$.

\begin{definicion}
	El espacio columna consiste en todas las combinaciones lineales de las columnas. Las combinaciones son todos los posibles vectores $Ax$. Estos llenan el espacio columna $C(A)$
\end{definicion}
``Ax=b tiene soluci\'on si y solo si b est\'a en el espacio columna de A.''

\subsection{El espacio nulo de A: Resolviendo Ax=0}
\begin{enumerate}
	\item El espacio nulo de A ``N(A)'' contiene todas las soluciones tales que Ax=0.
	\item La eliminaci\'on produce una matriz echelon U, y entonces una fila reducida R, con columnas pivots y columnas libres.
	\item Cada columna libre de U o R lleva a una soluci\'on especial. La variable libre se iguala a 1 y las otras variables libres a 0. La sustituci\'on por atr\'as resuelve Ax=0.
	\item La soluci\'on completa a Ax=0 es una combinaci\'on de las soluciones especiales.
	\item Si $n>m$ entonces A tiene al menos una columna sin pivot, dando una soluci\'on especial. Por lo que hay vectores x diferentes de cero en el espacio nulo de esta matriz A rectangular.
\end{enumerate}

\subsection{El rango y la forma reducida de fila}
\begin{enumerate}
	\item El rango (r) de A es el n\'umero de pivots despu\'es de la eliminaci\'on.
	\item El rango es la dimensi\'on del espacio columna. $r=dim(C(A))$
	\item Las columnas pivots no son combinaciones de otras columnas.
	\item Las columnas libres son combinaciones de otras columnas. Estas combinaciones son las soluciones especiales.
	\item Ax=0 tiene r pivots y n-r variables libres.
	\item La matriz del espacio nulo N contiene las n-r soluciones especiales. Entonces AN=0.
\end{enumerate}

\subsection{La soluci\'on completa de Ax=b}
X particular soluciona $Ax_p=b$.

X null soluciona $Ax_n=0$

La soluci\'on completa X= $x_p + x_n$.

\subsubsection{Full column rank (r=n)}
\begin{itemize}
	\item Todas las columnas de A son columnas pivot.
	\item No hay variables libres o soluciones especiales.
	\item El espacio nulo N(A) contiene solo el vector cero.
	\item Si Ax=b tiene soluci\'on (puede no tener) entonces es \'unica.
\end{itemize}

\subsubsection{Full row rank (r=m)}
\begin{itemize}
	\item Todas las filas tienen pivots, y R no tiene fila de cero.
	\item Ax=b tiene una soluci\'on por cada lado derecho b.
	\item El espacio columna es todo el espacio $R^m$.
	\item Existen n-r=n-m soluciones especiales en el espacio nulo de A.
\end{itemize}

\subsection{Independencia, bases y dimensi\'on}
Cada vector en el espacio es una combinaci\'on \'unica de los vectores bases.
\begin{definicion}
	Las columnas de A son linealmente independientes cuando la \'unica soluci\'on a Ax=0 es x=0. Ninguna otra combinaci\'on Ax de las columnas dan el vector cero.
\end{definicion}
\begin{definicion}
	La secuencia de vectores $V_1, V_2, \ldots , V_n$ es linealmente independiente si la \'unica combinaci\'on que genera el vector cero es $0V_1+0V_2+\ldots+0V_n$.
\end{definicion}
\begin{definicion}
	Un conjunto de vectores generan un espacio si las combinaciones lineales llenan el espacio.
\end{definicion}

\begin{definicion}
	El espacio fila de una matriz es el subespacio de $R^n$ generado por las filas.
	
	El espacio fila de A es $C(A^T)$. Esto es el espacio columna de $A^T$.
\end{definicion}

\begin{itemize}
	\item Los vectores bases son L.I y ellos generan el espacio.
	\item Los vectores $V_1,\ldots,V_n$ son bases para $R^n$ exactamente cuando ellos son las columnas de una matriz invertible nxn. Entonces $R^n$ tiene infinitas y diferentes bases.
	\item Las columnas pivots son una base para su espacio columna. Las filas pivots de A son una base para su espacio fila.
\end{itemize}
\begin{definicion}
	La dimensi\'on de un espacio es el n\'umero de vectores de la base.
\end{definicion}

\subsection{3.6 - Dimensiones de los cuatro subespacios}
\begin{enumerate}
	\item El espacio fila es $C(A^T)$, un subespacio de $R^n$.
	\item El espacio columna es $C(A)$, un subespacio de $R^m$.
	\item El espacio nulo es $N(A)$, un subespacio de $R^n$.
	\item El espacio nulo izquierdo es $N(A^T)$, un subespacio de $R^m$.
\end{enumerate}
Obs. Ver el big picture.

\subsubsection{Teorema fundamental del Algebra Lineal, Parte I}
$C(A)$ y $C(A^T)$ tienen la misma dimensi\'on r.

$N(A)$ y $N(A^T)$ tienen dimensiones n-r y m-r.

\section{Capitulo 4 Ortogonalidad}

\subsection{La ortogonalidad de los cuatro subespacios}
\begin{itemize}
	\item Subespacios V y W son ortogonales si cada v en V es ortogonal a cada w en W.
	\item V y W son complementos ortogonales si W contiene todos los vectores perpendiculares a V (y viceversa). Dentro $R^{n}$, las dimensiones de complementos V y W se suman a n.
	\item El espacio nulo N (A) y el espacio de la fila C ($A^{T}$) son complementos ortogonales, de Ax = 0. Del mismo modo N ($A^{T}$) y C (A) son complementos ortogonales.
	\item n vectores independientes en $R^{n}$ abarcarán (span) $R^{n}$.
	\item Cada x en $R^{n}$ tiene un componente $x_{n}$ en el espacio nulo y un componente $x_{r}$ en el  espacio fila.
\end{itemize}

\subsection{Proyecciones}
\begin{itemize}
	\item La proyección de b sobre a es $p=a\widehat{x}=a(a^{T}b/a^{T}a)$
	\item El rango de una matriz de proyección $P=aa^{T}/a^{T}a$ multiplica b para producir p.
	\item b proyectar sobre un subespacio deja e=b-p perpendicular al subespacio.
	\item Cuando A es de rango completo n, la ecuación $A^{T}A\widehat{x} = A^{T}b$ conduce a $\widehat{x}$ y $p = A\widehat{x}$
	\item La matriz de proyección $P=A(A^{T}A)^{-1}A^{T}$ tiene $P^{T}=P$ y $P^{2}=P$.
\end{itemize}
\subsection{Aproximaciones por mínimos cuadrados}
\begin{itemize}
	\item La solución por mínimos cuadrados $\widehat{x}$ ocurre cuando se minimiza $E=\left \| Ax-b \right \|^{2}$. Esta es la suma de los cuadrados de los errores en las m ecuaciones $(m > n)$.
	\item La mejor $\widehat{x}$ proviene de las ecuaciones normales $A^{T}A\widehat{x}=A^{T}b$.
	\item Para encajar m puntos por una línea b = C + Dt, las ecuaciones normales estan dadas por C y D.
	\item Las alturas de la mejor línea son $p=(p_{1},\ldots,p_{m})$. Las distancias verticales a los puntos de datos son los errores $e=(e_{1},\ldots,e_{m})$.
	\item Si tratamos de encajar m puntos por la combinación de $n<m$ funciones, las m ecuaciones Ax = b son generalmente insolubles. Las n ecuaciones $A^{T}A\widehat{x}=A^{T}b$ dan la solución de mínimos cuadrados.
\end{itemize}

\subsection{Base ortogonal y Gram-Schmidt}
\begin{itemize}
	\item Si los vectores ortonormales $q_{1},\ldots,q_{n}$ son las columnas de Q, entonces $q_{i}^{T}q_{j}=0$ y $q_{i}^{T}q_{i}=1$ se traducen en $Q^{T}Q=I$.
	\item Si $Q$ es cuadrada (una matriz ortogonal), entonces $Q^{T}=Q^{-1}$: transpuesta = inversa.
	\item La longitud de $Qx$ es igual a la longitud de $x$: $\left \| Qx \right \|=\left \| x \right \|$.
	\item La solución de mínimos cuadrados de $Qx=b$ es $\widehat{x} = Q^T b$. La proyección sobre el espacio columna que es abarcado por las $q$ es $P = QQ^{T}$.
	\item $Q$ también preserva los productos punto: $(Qx)^T (Qy) = x^T Q^T QY = x^T y$.
	\item Si $Q$ es cuadrada entonces $P = I$ y todos los $b=q_{1}(q_{1}^{T}b)+\ldots+q_{n}(q_{n}^{T}b)$.
	\item Gram-Schmidt produce vectores ortonormales $q_{1},q_{2},q_{3}$Q1, Q2, Q3 de ser independiente $a, b, c$. La factorización de la matriz es $A = QR=$(Q ortogonal) (R triangular).

	\item \textbf{El proceso de Gram-Schmidt:} Primero se elige $A=a$. La siguiente dirección $B$ debe ser perpendicular a $A$. \textbf{Se inicia con b y se substrae la proyección sobre $A$}. Ésto es:

	$B = b - \frac{A^T b}{A^T A}A$.

	$A$ es ortogonal a $B$. Siguiente paso:

	$C = c - \frac{A^T c}{A^T A}A - \frac{B^T c}{B^T B}B$.

	Finalmente, {\it dividir los vectores ortogonales $A,B,C...$ por sus longitudes.} Los vectores resultantes $q_1,q_2,...$ son ortonormales.


\end{itemize}

\section{Capitulo 5 Determinante}

\subsection{Propiedades de las determinates}

\begin{itemize}
	
	\item El determinante se define por $ det\ I=1 $, inversi\'on de signo y la linealidad en cada fila. 
	
	\item Despu\'es de de la eliminaci\'on det$ A $ es $ \pm $ (producto de los pivots).
	
	\item La determinante es exactamente cero cuando $ A $ es no invertible. 
	
	\item Dos propiedades notables son $ det\ AB=(det\ A)(det\ B) $ y $ det\ A^{T}=det\ A $. 
	
\end{itemize}

\subsection{Permutaciones y cofactores}

\begin{itemize}
	
	\item Sin intercambio de filas $ det\ A = (producto\ de\ pivots) $. En la esquina superior izquierda $ det\ A_{k} =$\textit{(producto de los primeros $ k $ pivots). }
	
	\item Cada t\'ermino en la gran f\'ormula (8) utiliza cada fila y columna de una vez. La mitad de los n! t\'erminos tienen signos + (cuando $ det\ P = +1 $) y la otra mitad tiene signo menos.
	
	\item El cofactor $ c_{ij} $ es $ (-1)^{i+j} $ veces la determinante m\'as pequeña que omite la fila $ i $ y columna $ j $ (debido al uso de la columna y fila $ a_{ij} $). 
	
	\item La determinante es el producto punto de cualquier fila de $ A $ con las filas del cofactor. Cuando la fila de $ A $ tiene muchos ceros, solo necesitamos unos pocos cofactores. 
	
	\end{itemize}
	
\subsection{Regla de Cramer, inversos y volumenes}

\begin{itemize}
	
	\item La regla de Cramer resuelve $ Ax=b $ por relaciones del tipo $ x_{1}=|B_{1}|/|A|=|ba_{1}...a_{n}|/|A| $.
	
	\item Cuando C es la matriz cofactor para $ A $, la inversa es $ A^{-1}=C^{T}/det\ A $.
	
	\item El volumen de una caja es $ |det\ A| $, cuando los bordes de la caja son la fila de $ A $.
	
	\item \'Area y volumen son necesarios para cambiar las variables en integrales dobles y triples.
	
	\item En $ \mathbb{R}^{3} $, el producto cruz $ u \times v $ es perpendicular a $ u $ y $ v $. 
	
\end{itemize}

\section{Capitulo 6 Valorespropios y vectorespropios}

\subsection{Introducci\'on a valorespropios}

\begin{itemize}
	\item $ Ax=\lambda x $ dice que el vectorpropio $ x $ mantiene la misma direcci\'on cuando se multiplica por $ A $. 
	
	\item $ Ax=\lambda x $ tambi\'en dice que $ det(A-\lambda I)=0 $. Esto determina $ n $ valores propios. 
	
	\item Los valores propios de $ A^{2} y A^{-1} $ son $ \lambda^{2} $ y $ \lambda^{-1} $, con igual vectorpropio. 
	
	\item La suma de $ \lambda's $ es igual a la suma de la diagonal principal de $ A $ (\textit{The trace}). El producto de $ \lambda's $ es igual a la determinante. 
	
	\item Matriz de proyecci\'on $ P $, motriz de reflexi\'on $ R $, matriz de rotaci\'on 90° $ Q $ tiene valorespropios especiales $ 1,\ 0,\ -1,\ i,\ -i $. Una matriz singular tiene $ \lambda=0 $. Una matriz triangular tiene $ \lambda's $ en su diagonal. 
	
\end{itemize}

\subsection{Diagonalizar una matriz}

\begin{itemize}
	\item Si $ A $ tiene $ n $ vectorespropios independiente $ x_{1},...,\ x_{n} $, estos entran en el espacio columna de $ S $.\\
	\[
	S^{-1}AS=\Lambda 
	\] 
	\[
	A=S\Lambda S^{-1}
	\] 
	
	\item Las potencias de $ A $ son $ A^{k}=S \Lambda^{k}S^{-1} $. Los vectores propios de $ S $ no se han modificado. 
	
	\item Los valores propios de $ A^{k} $ son $ (\lambda_{1})^{k},...,(\lambda_{n})^{k} $ en la matriz $ \Lambda^{k} $.
	
	\item La soluci\'on a $ u_{k+1}=Au_{k} $ empezando desde $ u_{0} $ es $ U_{k}=A^{k}u_{0}=S\Lambda S^{-1}u_{0} $: 
	\[
	u_{k}=c_{1}(\lambda)^{k}x_{1}+\cdots+c_{n}(\lambda)^{k}x_{n}
	\]
	siempre que 
	\[
	u_{0}=c_{1}x_{1}+\cdots+c_{n}x_{n}
	\]
	Esto muestran los pasos 1, 2, 3 ($ c's $ desde $ S^{-1}u_{0} $, $ \lambda^{k} $ desde $ \Lambda^{k} $ y $ x's $ desde $ S $).
	
	\item $ A $ es diagonalizable si cada valorpropio tiene suficientes vectorespropios (GM=AM). 
	
\end{itemize}

\subsection{Aplicaci\'on a ecuaciones diferenciales}

\begin{itemize}
	
	\item La ecuaci\'on $ u'=Au $ es lineal con coeficientes constantes, empezando desde $ u(0) $. 
	
	\item Su soluci\'on es por lo general una combinaci\'on de exponenciales, implicando cada $ \lambda $ y $ x $: 
	
	\textbf{Vectorespropios independientes}
	\[
	u(t)=c_{1}e^{\lambda_{1}t}x_{1}+\cdots+c_{n}e^{\lambda_{n}t}x_{n}
	\]
	
	\item Las constantes $ c_{1},...,c_{n} $ son determinadas por $ u(0)=c_{1}x_{1}+\cdots+c_{n}x_{n}=Sc $.
	
	\item $ u(t) $ se aproxima a cero (estabilidad) si cada $ \lambda $ tiene parte real negativa. 
	
	\item La soluci\'on es siempre $ u(t)=e^{At}u(0) $, con la matriz exponencial $ e^{At} $. 
	
	\item Ecuaciones con $ y'' $ se reducen a $ u'=Au $ por combinaci\'on de $ y' $ e $ y $ en $ u=(y,y') $.  
	
\end{itemize}

\subsection{Matrices simetricas}

\begin{itemize}

	\item La matriz sim\'etrica $ A $ tiene \textit{valorespropios real} y \textit{vectorespropios perpendiculares}.
	
	\item La diagonalizaci\'on se convierte a $ A=Q\Lambda Q^{T} $ con una matriz ortogonal $ Q $. 
	
	\item Todas las matrices sim\'etricas son diagonalizables, incluso con valores propios repetidos. 
	
	\item El signo del valorpropio coincide con el signo del pivots, cuando $ A=A^{T} $.  
	
	\item Toda matriz cuadrada puede ser ``triagonalizada" por $ A=QTQ^{-1} $. 
	
\end{itemize}

\subsection{Matriz defiida positiva}

\begin{itemize}
	
	\item Una matriz definida positiva tiene valoresprepios positivos y pivots positivos.
	
	\item Un test r\'apido esta dado por la determinante de la parte superior izquierda: $ a > 0 $ y  $ ac-b^{2} > 0 $. 
	
	\item La gr\'afica de $ x^{T}Ax $ es entonces un ``bowl" subiendo desde $ x=\textbf{0} $: 
	\[
	x^{T}Ax=ax^{2}+2bxy+cy^{2}
	\]
	es positivo excepto en $ (x,y)=(0,0) $.
	
	\item $ A=R^{T}R $ es autom\'aticamente positiva definida si $ R $ tiene columnas independientes. 
	
	\item La elipse $ x^{T}Ax=1 $ tiene sus ejes a lo largo de los vectores propios de $ A $. Con longitudes $ 1/\sqrt{\lambda} $.
	
\end{itemize}

\subsection{Matrices similares}

\begin{itemize}
	
	\item $ B $ es similar a $ A $ si $ B=M^{-1}AM $, para alguna matriz invertible $ M $. 
	
	\item Matrices similares tienen iguales valorespropios. Los vectorespropios se multiplican por $ M^{-1} $.
	
	\item Si $ A $ tiene $ n $ vectorespropios independiente entonces $ A $ es similar a $ \Lambda $ ($ M=S $). 
	
	\item Cada matriz es similar a una matriz de Jordan $ J $ (que tiene $ A $ como su parte diagonal). $ J $ tiene un bloque para cada vectorpropio y 1's para los vectorespropios faltantes. 
	
\end{itemize}

\subsection{Descomposici\'on en valor singular (SVD)}

\begin{itemize}
	
	\item El factor SVD de $ A $ en $ V\Sigma V^{T} $ de $ r $ valores singulares $ \sigma_{1}\geq ... \geq \sigma_{r} > 0 $.
	
	\item Los n\'umeros $ \sigma^{2}_{1},..., \sigma^{2}_{r} $ son los vectorespropios distintos de cero de $ AA^{T} $ y $ A^{T}A $.
	
	\item La columna ortonormal de $ U $ y $ V $ son vectorespropios de $ AA^{T} $ y $ A^{T}A $. 
	
	\item Esas columnas tiene  bases ortonormales para los cuatro espacios fundamentales de $ A $. 
	
	\item Esas bases diagonalizan la matriz: $ Av_{i}=\sigma_{1}u_{i} $ para $ i \leq r $. Esto es $ AV=U\Sigma $. 
	
	
\end{itemize}

Este material fue elaborado por:\\
 
	Esteban Lailla stban06@gmail.com\\
	Juan Carlos Miranda juancarlosmiranda81@gmail.com\\
	Julio Mello prof.juliomello@gmail.com\\
	Pastor Enmanuel P\'erez-Estigarribia peperez.estigarribia@gmail.com \\
	Luis G. Moré lmore@pol.una.py

%%%%%%%%%%%%%%%%%%%%%%%%%%%%%%%%%%%%%%%%%%%%%%%%%%%%%%%%%%%%%%%%%%%%%%%%%%%%%%%%%%%%%%%%%%%%%%%%%%%%%%%%%%%%%%%%%%%%%%%%%%%%%
\renewcommand{\refname}{Bibliograf\'ia consultada}
\begin{thebibliography}{99}
	
	\bibitem{DU} GILBERT STRANG.
	\emph{INTRODUCTION TO LINEAR ALGEBRA (Fourth Edition)},
	ISBN 978-0-9802327-2-1
	
	
\end{thebibliography}

\end{document}